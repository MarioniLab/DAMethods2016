\documentclass{article}
\usepackage{graphicx}
\usepackage[margin=2.5cm]{geometry}
\usepackage[labelfont=bf]{caption}
\usepackage{subcaption}
\usepackage{color}
\usepackage{xr}
\usepackage{xcite}
\usepackage{amsmath}
\usepackage[euler]{textgreek}

\renewcommand{\textfraction}{1.0}
\renewcommand{\floatpagefraction}{.9}
\newcommand\revised[1]{\textcolor{red}{#1}}
\renewcommand{\topfraction}{0.9}    % max fraction of floats at top
\renewcommand{\bottomfraction}{0.8} % max fraction of floats at bottom
\renewcommand{\textfraction}{0.07}  % allow minimal text w. figs

\makeatletter 
\renewcommand{\fnum@figure}{Supplementary \figurename~\thefigure}
\renewcommand{\fnum@table}{Supplementary \tablename~\thetable}
\makeatother

%\renewcommand{\thefigure}{S\@arabic\c@figure} 
%\renewcommand{\thetable}{S\@arabic\c@table} 

\externaldocument{description}
\externalcitedocument{description}

\usepackage{url}
\urlstyle{same}

\begin{document}

\begin{titlepage}
\vspace*{3cm}
\begin{center}

{\LARGE
Testing for differential abundance in mass cytometry data
\par}

\vspace{0.75cm}

{\Large 
    \textsc{Supplementary Materials}
\par
}
\vspace{0.75cm}

\large
by

\vspace{0.75cm}
Aaron T. L. Lun$^{1}$, Arianne C. Richard$^{1,2}$ and John C. Marioni$^{1,3,4}$

\vspace{1cm}
\begin{minipage}{0.9\textwidth}
\begin{flushleft} 
$^1$Cancer Research UK Cambridge Institute, University of Cambridge, Li Ka Shing Centre, Robinson Way, Cambridge CB2 0RE, United Kingdom \\[6pt]
$^2$Cambridge Institute for Medical Research,  University of Cambridge, Wellcome Trust/MRC Building, Hills Road, Cambridge CB2 0XY, United Kingdom \\[6pt]
$^3$EMBL European Bioinformatics Institute, Wellcome Genome Campus, Hinxton, Cambridge CB10 1SD, United Kingdom \\[6pt]
$^4$Wellcome Trust Sanger Institute, Wellcome Genome Campus, Hinxton, Cambridge CB10 1SA, United Kingdom \\[6pt]
\end{flushleft}
\end{minipage}

\vspace{1.5cm}
{\large \today{}}

\vspace*{\fill}
\end{center}
\end{titlepage}

\section{Defining the hypersphere radius for cell counting}

\subsection{Formulating an expression for the radius}
Consider a subpopulation where the intensities of each marker are distributed around some mean intensity with standard deviation $r_0$.
%The center of the subpopulation is defined by the set of expected intensities.
The expectation of the squared Euclidean distance between any two cells in this subpopulation is $2r_0^2M$.
Now, construct hyperspheres centred at each of the cells in the subpopulation.
Setting the hypersphere radius as $r=r_0\sqrt{2M}$ ensures that most hyperspheres will contain a large proportion of cells from the same subpopulation.
Indeed, a simple simulation with Normally-distributed intensities for 30 markers indicates that the majority of hyperspheres contain more than 50\% of the subpopulation.

% To wit, E[(Xa - Xb)^2] = E(Xa^2) + E(Xb^2) - 2E(XaXb)
%                        = r0^2 + u^2 + r0^2 + u^2 - 2u^2
%                        = 2r0^2
% And then sum this over all markers. We can demonstrate with some code:
%   a <- matrix(runif(30000, 0, r), ncol=30)
%   out <- as.matrix(dist(a))
%   sqrt(var(a[,1])) * sqrt(30) * sqrt(2)
%   mean(out)
% The same works with normally distributed intensities:
%   r <- 1
%   a <- matrix(rnorm(30000, 1, r), ncol=30)
%   out <- as.matrix(dist(a))
%   sum(colSums(out <= r*sqrt(2*ncol(a))) >= nrow(a)/2)

We further assume that the expression of a marker will differ by up to 10-fold across functionally similar cells, due to technical noise or biological variability within the same subpopulation.
For typical analyses involving transformed intensities on the $\log_{10}$-scale, an interval of length 1 will span an order of magnitude in marker expression.
When centred around the mean, this interval should contain most of the intensity distribution for a single subpopulation.
According to Chebyshev's inequality, the interval defined by a distance of $r_0\sqrt{2}$ from the mean will contain at least 50\% of the values from an arbitrary distribution.
We equate these interval lengths to obtain a rough \revised{upper bound} for the standard deviation of the intensities, i.e., $2r_0\sqrt{2} = 1$.
Applying this to the radius yields $r=0.5\sqrt{M}$, meaning that each hypersphere will be able to include cells with 10-fold differences in expression for one or more markers.
Thus, each hypersphere will contain enough counts for further analysis, even when the intensities vary across an order of magnitude.

%(A lower proportion of cells in the subpopulation will be counted if the hypersphere is not centred at the subpopulation centre.
%This will result in reduced power to detect differential abundance for that particular hypersphere.
%However, as long as there is one hypersphere that is centred at the subpopulation centre, then the potential to reject the null hypthesis for the corresponding subpopulation will not be lost.
%Recall that each hypersphere is centred at an observed cell, so any subpopulation in the data set with a reasonable number of cells ($>100$) is likely to be captured by an appropriate hypersphere near the subpopulation centre.)

% A consequence of setting $r$ like this is that, in any one dimension, it is possible to count cells that are more than $r_0$ from the centre of the hypersphere.
% This is inevitable in multi-dimensional space, where you need to account for scatter in dimensions other than the one of immediate interest.
% Setting $r$ any smaller would fail to count the majority of cells derived from the same subpopulation.
% Indeed, a more straightforward definition of each subspace would be to only count cells that are less than $r_0$ away from the centre in every dimension.
% This defines the subspace as an $M$-dimensional box with sides of length $2r_0$.
% However, the use of boxes is not practical as the probability of cells falling into a box will decrease with increasing $M$. 
% This results in near-zero counts (or counts of 1, for boxes centred on an existing cell).
% Hyperspheres are more suitable at large $M$, which motivates their use above.

We verify the choice of formulation of $r$ by examining the distance between each cell $x$ and its neighbours in the first sample from each time course.
The nearest neighbours of $x$ are identified in the full $M$-dimensional space and represent other cells in the same subspace as $x$.
As the number of markers increases, the distance to the neighbours increases at a rate that is roughly consistent with the square root function (Supplementary Figure~\ref{fig:radius}).
This justifies the use of the relation $r\propto\sqrt{M}$.
With increasing $M$, the radius will also increase such that a hypersphere centred at $x$ will still contain the neighbours of $x$, i.e., sensitivity of counting for cells in the same subspace is preserved.
In contrast, cells in other subspaces will cease to be assigned to the hypersphere as they become separated from $x$ on other markers, i.e., specificity of counting is improved.
   
For the Oct4-GFP and Nanog-Neo time courses, the radius is consistent with the median distances from each cell to its 10\textsuperscript{th} nearest neighbour.
This means that around 10 cells from the subspace will be assigned to each hypersphere when $r_0\sqrt{2}=0.5$.
Combined with counts from the other samples in the time course, this provides enough information to reliably detect DA hyperspheres.
In contrast, the radius in the Nanog-GFP time course is consistent with the distance of each cell to its 1\textsuperscript{st} nearest neighbour.
This is because the first sample in the Nanog-GFP time course contains fewer than 3000 cells, while samples in the other time courses contain over 10000 cells.
The sparsity of cells in this sample results in fewer neighbours at any given distance.
For experiments with few cells across all samples, a larger $r_0$ may be required to obtain sufficient counts for further analysis.
(Note that we keep $r_0\sqrt{2}=0.5$ for Nanog-GFP, as the other samples in the time course contain 5-10-fold more cells.
Thus, counts will be large enough to detect differences later on.)

\subsection{Additional remarks on radius choice}
\label{sec:additionalradius}

\begin{color}{red}
The suitability of the default setting of $r_0\sqrt{2}=0.5$ can be assessed for each data set by examining the distance from each cell to its nearest neighbours.
Supplementary Figure~\ref{fig:nvd} shows these distances for each cell in the Oct4-GFP reprogramming time course.
Here, the default radius corresponds to the median of the distances from each cell to its 15\textsuperscript{th} nearest neighbour.
This means that half of all hyperspheres will contain more than 15 cells, which is large enough for downstream hypothesis testing.
In contrast, a smaller value of the radius (e.g., $r_0\sqrt{2}=0.4$) would not even include the nearest neighbour for each cell on average.
This would lead to a count of 1 for most hyperspheres, which is not sufficient for further statistical analysis.

A consequence of working in high-dimensional space is that small changes to the radius will result in large changes to the counts.
For example, in an experiment consisting of 30 markers, increasing the radius by 10\% will increase the hypersphere volume (and the potential number of counted cells) by over 17-fold. 
In this respect, the analysis is quite sensitive to the choice of radius.
However, a more relevant assessment of sensitivity is that with respect to changes in the counts per hypersphere.
We repeated the DA analyses using radii values that approximately halved or doubled the median number of counted cells in each hypersphere relative to the default (Supplementary Figure~\ref{fig:nvd}).
We observed that most of the DA hyperspheres from the default analysis were still detected in the repeated analyses (Supplementary Table~\ref{tab:param}).
Over 60\% of the DA hyperspheres were recovered with a smaller radius, while using a larger radius detected almost all of them (and an extra 35\%).
These differences are expected because larger hyperspheres have greater counts and more evidence to reject the null hypothesis, albeit at the cost of spatial resolution.
Nonetheless, the general similarity in the results indicates that the analysis is robust to substantial changes in the size of the counts.
\end{color}

\revised{An obvious question is whether an ``optimal'' value of $r_0$ can be obtained for any given data set.}
This \revised{value} depends on the variability of marker intensities in \revised{each} subpopulation.
For example, setting $r_0$ to the standard deviation of the marker intensities \textit{for a single subpopulation} allows more cells associated with that subpopulation to be counted into the hypersphere, while reducing the chance of counting cells from adjacent subpopulations.
However, the standard deviation of each subpopulation is not easy to estimate empirically, as we do not know the true number of subpopulations in the $M$-dimensional space.
Indeed, a single optimal value may not exist, e.g., if different subpopulations or markers have different standard deviations.

\revised{A suboptimal} choice of $r_0$ is not a critical concern for a DA analysis.
A value of $r_0$ that is too small will reduce power to detect a DA subspace as the counts are too low.
A value that is too large will also reduce power as cells from non-DA subspaces are included in the count for each hypersphere.
In both cases, power is affected rather than error rate control.
While loss of power is undesirable, the analysis will still be valid as any discoveries (that are made in spite of the diminished power) can be trusted.
We note that ``contamination'' of a non-DA hypersphere with cells from DA subspaces is also possible for large $r_0$, which could lead to the detection of the former as a false positive.
However, the effect of this contamination is mitigated by the fact that the position of each hypersphere is defined using the median intensities.
Any contamination that substantively changes the counts for a hypersphere will also shift its position towards the differential subspace.
Thus, a contaminated hypersphere that is erroneously classified as \revised{DA} will be assigned to or near the differential subspace (based on its position), rather than a non-differential subspace (based on its centre).
This reduces the detection rate of false positives in the non-differential subspace.

\begin{color}{red}
To demonstrate this effect, we set up a simulation where we examined the effect of increasing the radius on the position of each hypersphere (Supplementary Figure~\ref{fig:radius_schematic}).
When larger radii were used, hyperspheres centred on cells from a non-DA subpopulation had larger log-fold changes in abundance (Supplementary Figure~\ref{fig:radius_position}).
This is consistent with increasing contamination from cells in the neighbouring DA subpopulation.
However, the positions of the affected hyperspheres were also shifted towards the DA subpopulation.
In most simulation scenarios, this shift was sufficiently large that the hyperspheres with large log-fold changes ($>1$) were not distinguishable from hyperspheres centred on cells from the DA subpopulation itself.
The only exception occurs when the size of the DA subpopulation was reduced, such that some of the non-DA hyperspheres have large log-fold changes but lie outside the subpopulation boundaries.
Even in this case, a substantial shift in position towards the DA subpopulation was observed, mitigating the effect of misinterpreting these hyperspheres as part of a differential subspace.
Using very large hypersphere radii also results in a decrease in the log-fold changes for hyperspheres centred on cells from the DA subpopulation.
This is consistent with loss of power when cells from non-DA subspaces are included in the counts.
\end{color}

\section{Handling composition effects in the counts}
Composition effects are caused by large changes in cell abundance in one or more subspaces that alter the total counts.
This results in differential proportions being observed in otherwise non-differential hyperspheres.
From a mathematical perspective, this is not incorrect as the proportions \textit{do} change when the total counts are altered.
Nonetheless, detection of such subspaces may not be biologically relevant, so composition effects should ideally be removed prior to further analysis.
Unfortunately, conventional strategies for normalizing these effects (from RNA-seq data analysis \cite{robinson2010scaling}) are not applicable here.
If these methods were applied to the hypersphere counts, they would assume that most hyperspheres are not differentially abundant.
This is unlikely to be true in many settings, e.g., due to large-scale changes upon stimulation or activation.

Rather, alternative strategies are required to mitigate composition effects.
The simplest approach is to gate out any high-abundance differential subpopulation.
The total count can then be calculated from the remaining cells, which avoids introducing composition effects from the differential subpopulation.
Identification of problematic subpopulations can be done before the DA analysis based on existing knowledge, or afterwards based on the top DA hyperspheres.
For example, in a mixed population of T and B cells, one can gate on CD3 or CD19 to isolate T or B cells respectively.
This avoids detecting changes in T-cell subpopulations due to changes in B-cell abundance, and vice versa.
Note that this is only necessary for DA subpopulations with many cells, as changes in small subpopulations will not have a substantial effect on the total count.

Another approach is to test for differential abundance against a minimum log-fold change threshold.
This avoids detecting small changes in abundance caused by composition effects.
Such changes may be statistically significant but are unlikely to be biologically relevant \cite{mccarthy2009treat}, and are subsequently ignored.
While this is a more general approach than gating, it only protects against small composition effects -- large changes in the dominant subpopulation may induce large changes in abundance across all hyperspheres.

\section{Controlling the spatial FDR with weights}
Let each null hypothesis $i$ be associated with a $p$-value $p_{(i)}$ and a frequency weight $f_{(i)}$.
Assume that there are $n$ null hypotheses, ordered such that $p_{(1)} < p_{(2)} < ... < p_{(n)}$.
To control the FDR at some threshold $\alpha$, a weighted BH procedure is applied \cite{benjamini1997multiple} to reject any null hypothesis where the $p$-value is less than
\[
    T = \max\left\{ p_{(i)} : p_{(i)} \le \alpha \frac{\sum_{l=1}^{i} f_{(l)}}{\sum_{l=1}^{n} f_{(l)}} \right\}  \;.
\]
%Intuitively, this can be understood as treating the weighted hypothesis $i$ as a collection of $f_{(i)}$ unweighted hypotheses, and applying the standard BH method to the total set of unweighted hypotheses across all $i$.
%(This reasoning is still applicable for non-integer $f_i$, as such values can be scaled up to yield values that are arbitrarily close to integers.
%The calculation of $T$ is unaffected as any scaling will cancel out.
%Moreover, while the unweighted hypotheses in each collection are completely dependent, this should not compromise FDR control as the BH method is robust to dependencies \cite{reiner2003identifying,kim2008effects}.)

To control the spatial FDR, we apply the weighted BH method to the hypersphere statistics.
Each hypersphere corresponds to one null hypothesis, while its frequency weight is defined as the reciprocal of its local density.
Here, a decision must be made regarding the choice of kernel density estimator and bandwidth.
For the latter, we compute the distance from each hypersphere position to its 50\textsuperscript{th} nearest neighbour.
The bandwidth is defined as the median of this distance across all hyperspheres.
This ensures that, on average, around 50 neighbours will be available to stably calculate the local density for each hypersphere.
We also use a tricube kernel to provide some robustness to the choice of bandwidth (Supplementary Table~\ref{tab:param}).
This gives more weight to closer neighbours while reducing the influence of cells that fall just inside the bandwidth.

It is worth noting that we control the FDR in the expected set of representative hyperspheres, rather than the expectation of the FDR with respect to all possible sets of randomly sampled representatives.
Formally speaking, control of the FDR usually refers to controlling the latter value.
The FDR across the expected set is used in our analysis because it is simpler to compute.
It also approaches the expected FDR for small partitions, as strong correlations betwen hyperspheres in the same partition mean that the expected set will have a similar (frequency-weighted) distribution of $p$-values as any instance of the sampled set.
A similar issue arises from the fact that visualization of the differential subspaces is performed in low-dimensional space to facilitate interpretation.
There is no guarantee that the FDR across, say, pixels in 2D space is equal to the spatial FDR computed in the original $M$-dimensional space.
Nonetheless, simulation testing in Figure~\ref{fig:fdr} indicates that the FDR in the reduced space is successfully controlled along with the spatial FDR.

\newcommand{\hi}{\textsuperscript{high}}
\newcommand{\lo}{\textsuperscript{low}}

\section{Annotating subpopulations in the Oct4-GFP time course}
To annotate Figure~\ref{fig:oct4} in the main text, we examined the marker intensities of each characterised subpopulation in Figure~3D of Zunder \emph{et al.} \cite{zunder2015continuous} and identified the cluster with the most similar intensities in our Supplementary Figure~\ref{fig:oct4_markers}.
The important markers for each subpopulation are listed below:
\begin{itemize}
    \item MEFs were identified as Thy1\hi{}mEF-SK4\hi{}Cd140a\hi{} and Oct4\lo{}Sox2\lo{} along with lower expression of Klf4 relative to other subpopulations.
        This was visually supported by the fact that the same cells were pS6\hi{}\textbeta-Catenin\hi{}I\textkappa{}B\textalpha\hi{} in the original figure.
    \item The OSKM non-expressing population was identified as Oct4\lo{}Sox2\lo{} with lower Klf4.
        In addition, a majority of cells exhibited lower Thy1 and Cd140a expression compared to the neighbouring MEF population.
        This was further supported by low expression of pS6 in a majority of these cells, as well as reduced \textbeta-Catenin and I\textkappa{}B\textalpha{} expression and higher p53 expression relative to MEFs.
%    \item The transition between MEFs and OSKM non-expressing cells was identified as pS6\hi{}, with moderate Thy1, mEF-SK4 and Cd140a expression.
%\textbeta-Catenin and I\textkappa{}B\textalpha{} levels were lower than in MEFs.
    \item A small subset of MEFs in Figure~\ref{fig:oct4} exhibited very high Cd140a expression.
This represents OSKM non-expressing cells that reverted to a MEF-like endpoint state after doxycycline withdrawal.
    \item The Oct4\hi{}Klf4\hi{} population was identified as it was named, supported by high expression of Sox2.
    \item The SC4-like population was identified as Klf4\hi{}Oct4\hi{}Cd73\hi{}, additionally supported by high expression of Ki67 and low expression of Sox2.
    \item Cells undergoing mesenchymal-epithelial transition were identified as EpCAM\hi{}, supported by high expression of both \textbeta-Catenin and Oct4.
    \item Ki67\lo{} reverting cells were identified as Ki67\lo{}Cd73\hi{}.
        This annotation was supported by high expression of mEF-SK4 in parts of the population.
    \item The Nanog\hi{} trajectory in the original study was identified as a Nanog-intermediate continuum from Nanog\lo{} to Nanog\hi{} cells.
        This manifests in Supplementary Figure~\ref{fig:oct4_markers} as several Nanog-intermediate subpopulations close to the Lin28\hi{} and ESC-like subpopulations.
        The annotation was supported by high expression of EpCAM and SSEA1, as well as higher pS6 expression in some parts of the trajectory.
    \item ESC-like cells were identified as the population with the highest expression of Nanog and high expression of SSEA1 and Lin28.
        This was supported by high levels of H3K9ac, H4Kac, Cd54, pSrc and pERK.
    \item Lin28\hi{} cells were identified as named, supported by high expression of Cd24.
They were further distinguished from ESC-like cells by having higher expression of Cd140a and lower expression of Nanog.
    \item The Ki67\hi{} population was identified as named, supported by lower expression of Oct4.
    \item The mixed 4F population from the original study was highly heterogeneous and difficult to characterize.
We identified it as mostly Klf4\lo{}, containing subpopulations with intermediate c-Myc expression and both high and low Sox2 expression.
However, we were unable to identify the Oct4\lo{} subpopulations.
\end{itemize}
Many of these subpopulations were further resolved into distinct subsets based on specific markers such as IdU.
We also identified subpopulations that were not explicitly classified by Zunder \emph{et al.}
This includes a potentially apoptotic population with cleaved Caspase-3;
    a SC4-like subpopulation with phosphorylated STAT3, AMPK and PLK1;
    and a small set of MEF-like/mixed 4F cells that express Thy1, Sox2 and Oct4.
%   and an Oct\hi{}Klf4\hi{} population of cells expressing higher levels of Cyclin B1.


%%%% FIGURES BEGIN HERE %%%%
\newpage

\begin{figure}[p]
    \begin{center}
        \includegraphics[width=0.32\textwidth]{../real/neighbours/Cytobank_43324_4FI.pdf}
        \includegraphics[width=0.32\textwidth]{../real/neighbours/Cytobank_43324_NG.pdf}
        \includegraphics[width=0.32\textwidth]{../real/neighbours/Cytobank_43324_NN.pdf}          
    \end{center}
    \caption{
        Euclidean distance from each cell to its nearest neighbours as a function of the number of markers.
        For each cell in the first sample of each time course in the MEF reprogramming study, its nearest neighbours were identified in the full (36-dimensional) space.
        A subset of 9-25 markers were randomly chosen and the distance to each nearest neighbour was recalculated for each cell in the reduced space.
        The 10\textsuperscript{th} nearest neighbour was used for Oct4-GFP and Nanog-Neo, while the 1\textsuperscript{st} nearest neighbour was used for Nanog-GFP.
        Each point represents the \revised{mean} distance across all cells for a given number of markers, while the error bar represents the \revised{sample standard deviation across cells}.
        This was also repeated using all 36 markers.
        The red line marks the hypersphere radius that is used for each number of markers.
    }
    \label{fig:radius}
\end{figure}

\begin{figure}[p]
    \begin{center}
        \includegraphics[width=0.6\textwidth]{../real/neighbours/nvd_Cytobank_43324_4FI.pdf}
    \end{center}
    \caption{The \revised{dependence} of the hypersphere radius (based on the choice of $r_0\sqrt{2}$) on the cell count for each hypersphere in the Oct4-GFP time course.
        For each hypersphere \revised{centred on a cell}, the radius required to include a \revised{certain number of nearest neighbours} is computed.
        The distribution of radii across all hyperspheres is shown as a boxplot for each \revised{neighbour}.
        The red line represents the \revised{default} $r_0\sqrt{2}=0.5$.
        \revised{The blue lines mark the values of $r_0\sqrt{2}$ corresponding to median distances for 8 and 30 neighbours.}
    }
    \label{fig:nvd}
\end{figure}

\begin{table}[btp]
\caption{Effect of changes to the hypersphere radius on the DA analysis.
The number of hyperspheres retained after filtering is shown, along with the number detected with significant differences at a spatial FDR of 5\%.
In each altered analysis, the number of DA hyperspheres gained or lost relative to the original analysis is reported.
The effect of altering the bandwidth (BW) for spatial FDR calculation was also tested, by defining the bandwidth using the 20\textsuperscript{th} (smaller) and 100\textsuperscript{th} nearest neighbour (larger).
}
\label{tab:param}
\begin{center}
\begin{tabular}{l l r r r r r}
\hline
\textbf{Dataset} & \textbf{Statistic} & \textbf{Original} & \multicolumn{2}{c}{\textbf{Value of $r_0\sqrt{2}$}} & \multicolumn{2}{c}{\textbf{Bandwidth}} \\
                 &                    &          & \makebox[0.4in][r]{\textit{0.4}} & \makebox[0.6in][r]{\textit{0.6}} 
                                                 & \makebox[0.6in][r]{\textit{Smaller}} & \makebox[0.6in][r]{\textit{Larger}} \\
\hline
Oct4-GFP & Total & 7720 & 4984 & 10799 & 7720 & 7720 \\
 & Significant & 7416 & 4837 & 10242 & 7418 & 7414 \\
 & Gained & - & 35 & 2916 & 2 & 0 \\
 & Lost & - & 2614 & 90 & 0 & 2 \\
\hline
Nanog-GFP & Total & 6297 & 4137 & 8700 & 6297 & 6297 \\
 & Significant & 5947 & 3917 & 8291 & 5947 & 5944 \\
 & Gained & - & 63 & 2389 & 0 & 0 \\
 & Lost & - & 2093 & 45 & 0 & 3 \\
\hline
Nanog-Neo & Total & 22043 & 15025 & 28944 & 22043 & 22043 \\
 & Significant & 21532 & 14663 & 28271 & 21532 & 21532 \\
 & Gained & - & 53 & 6809 & 0 & 0 \\
 & Lost & - & 6922 & 70 & 0 & 0 \\
\hline
\end{tabular}
\end{center}
\end{table}

\begin{color}{red}
\begin{figure}[tbp]
    \begin{center}
        \includegraphics[width=0.8\textwidth,trim=10mm 50mm 10mm 50mm,clip]{../simulations/radius_setup.pdf}
    \end{center}
    \caption{Schematic of the simulation design for examining the effect of increasing the hypersphere radius.
        Here, the experimental design consists of two samples from different conditions and 30 markers.
        Two subpopulations are present in the high-dimensional space, one of which is DA between samples (red) and the other is not (grey).
        Cells from each subpopulation (10000 each) are uniformly distributed within a 30-dimensional sphere of radius (referred to as ``size'', above) of length $0.5\sqrt{30}$, centred at the subpopulation mean (crosses).
        The means of the two subpopulations are separated by a distance of 0.5 in each dimension.
        Each hypersphere's position is calculated by taking the median in each dimension across all cells in the hypersphere.
        Note that the radii of the hyperspheres and subpopulations need not be identical.
    }
    \label{fig:radius_schematic}
\end{figure}

\begin{figure}[tbp]
    \begin{center}
        \includegraphics[width=\textwidth]{pics/Figure_RadiusSim.pdf}
    \end{center}
    \caption{Simulation results demonstrating the effect of expanding the radius on the median-based position of each hypersphere.
        For each hypersphere, the distance between its position and the mean of the DA subpopulation was computed and plotted against the log-fold change in abundance between samples.
        Hyperspheres were coloured based on whether they were centred on cells from the DA (red) or non-DA (grey) subpopulations.
        The vertical dashed line marks the distance corresponding to the size of the DA subpopulation, while the horizontal dashed line represents a $\log_2$-fold change of 1.
        This process was repeated to create plots for a range of increasing hypersphere radii and for different simulation scenarios.
        Scenario 1 refers to the default scenario described in Supplementary Figure~\ref{fig:radius_schematic}, while scenario 2 involves decreasing the number of cells in the DA subpopulation to 5000; scenario 3 increases the distance between subpopulations to 0.8 in all dimensions; and scenario 3 decreases the size of the DA subpopulation to $0.2\sqrt{30}$.
    }
    \label{fig:radius_position}
\end{figure}
\end{color}{red}


\begin{figure}[p]
    \begin{center}
    \includegraphics[width=0.6\textwidth]{../real/analysis/pics/dispersions.pdf}
    \end{center}
    \caption{NB dispersion estimates for all hyperspheres in each time course of the MEF reprogramming data set, plotted against the average number of cells across all samples in each hypersphere.}
\end{figure}

\begin{figure}[p]
    \begin{center}
        \includegraphics[width=0.6\textwidth]{../simulations/FDR_setup.pdf}
    \end{center}
    \caption{Pixel-based partitioning of hypersphere positions in a simulation based on the Oct4-GFP time course.
        Detected hyperspheres were defined using the na\"ive BH method to nominally control the FDR at 5\%.
        The two-dimensional space was defined using the first two principal components after applying PCA on the detected hypersphere positions.
        The space was partitioned into pixels of width 0.5.
        Red pixels contain hyperspheres that are truly differential, while grey pixels contain only non-DA hyperspheres (i.e., false positives).
        For the latter, the depth of colour is proportional to the number of hyperspheres.
    }
\end{figure}

\newcommand{\bigfigopt}[1]{\includegraphics[width=\textwidth,draft]{#1}}

\begin{figure}[p]
    \begin{center}
    \bigfigopt{../real/analysis/pics/Cytobank_43324_4FI_markers.png}
    \end{center}
    \caption{
        Marker intensities of differentially abundant subpopulations in the Oct4-GFP time course, detected at a spatial FDR of 5\%.
        Each plot corresponds to a marker while each point represents a hypersphere, coloured according to its median intensity for the corresponding marker.
        The boundaries of the colour range for each marker were set to the 1\textsuperscript{st} and 99\textsuperscript{th} percentiles of the intensities for all cells.
    }
    \label{fig:oct4_markers}
\end{figure}

\begin{figure}[p]
    \begin{center}
    \includegraphics[width=0.6\textwidth]{../real/analysis/pics/Cytobank_43324_NG_logFC.png}
    \end{center}
    \caption{
        Differentially abundant subpopulations in the Nanog-GFP time course, detected at a spatial FDR of 5\%.
        Each point represents a hypersphere and is coloured according to its log-fold change in abundance over time.
        Grey points are hyperspheres with significant but non-linear changes in abundance.
    }
\end{figure}

\begin{figure}[p]
    \begin{center}
    \bigfigopt{../real/analysis/pics/Cytobank_43324_NG_markers.png}
    \end{center}
    \caption{
        Marker intensities of differentially abundant subpopulations in the Nanog-GFP time course, detected at a spatial FDR of 5\%.
        Each plot corresponds to a marker while each point represents a hypersphere, coloured according to its median intensity for the corresponding marker.
    }
\end{figure}

\begin{figure}[p]
    \begin{center}
    \includegraphics[width=0.6\textwidth]{../real/analysis/pics/Cytobank_43324_NN_logFC.png}
    \end{center}
    \caption{
        Differentially abundant subpopulations in the Nanog-Neo time course, detected at a spatial FDR of 5\% and coloured according to the log-fold change in abundance over time.
    }
\end{figure}

\begin{figure}[p]
    \begin{center}
    \bigfigopt{../real/analysis/pics/Cytobank_43324_NN_markers.png}
    \end{center}
    \caption{
        Marker intensities of differentially abundant subpopulations in the Nanog-Neo time course, detected at a spatial FDR of 5\%.
        Each plot corresponds to a marker while each point represents a hypersphere, coloured according to its median intensity for the corresponding marker.
    }
\end{figure}

\begin{figure}[p]
    \begin{center}
        \includegraphics[width=0.6\textwidth]{../real/analysis/pics/novel_timecourse.pdf}
    \end{center}
    \caption{
        Cell abundance for the novel SC4-like subpopulation with active STAT3, AMPK and PLK1 signalling, as a function of time in the Oct4-GFP reprogramming time course.
        Abundances are shown as a percentage of the total number of cells in each sample.
        Each line represents the abundance for each hypersphere representing the subpopulation, while the black line is the average across all hyperspheres.
    }
\end{figure}

\begin{figure}[p]
\begin{center}
\includegraphics[width=0.49\textwidth]{../simulations/cluster_setup.pdf}
\includegraphics[width=0.49\textwidth]{../simulations/plot_cluster.pdf}
\end{center}
\caption{Relative performance of cluster-based DA analyses in simulated data.
    Left: a PCA plot to illustrate the simulation design.
    A non-DA continuum of cells (grey) is present, in which DA subpopulations exclusive to the first (blue) or second conditions (red) are nested.
    Right: detection frequencies for each of the two DA subpopulations across 20 simulation iterations.
    In each iteration, DA analyses were performed using a cluster-based approach with varying numbers of clusters $k$ or with hyperspheres.
}
\label{fig:clustersim}
\end{figure}

\begin{figure}[p]
\begin{center}
\includegraphics[width=0.8\textwidth]{../real/clustering/Cytobank_43324_4FI_clusters.png}
\end{center}
\caption{DA clusters from a cluster-based analysis of the Oct4-GFP time course, mapped onto the $t$-SNE plot of DA hyperspheres.
Each coloured point represents the centre of a DA cluster that was detected at a FDR of 5\%.
Detected clusters are shown from several analyses performed with varying numbers of clusters $k$.
The colour of each point is based on the log-fold change per day in the cell abundance of the corresponding cluster.
DA hyperspheres are shown as a grey outline for comparison.
}
\end{figure}

\end{document}
